\documentclass{article}
\usepackage[a4paper, margin=0.6in]{geometry}
\usepackage{blindtext}
\usepackage{microtype}
\usepackage{setspace}
\usepackage{sectsty}
\usepackage{titlesec}
\usepackage{enumitem}
\usepackage{hyperref}
\usepackage{color}
\titlespacing\section{0pt}{12pt plus 4pt minus 2pt}{0pt plus 2pt minus 2pt}
\titlespacing\subsection{0pt}{12pt plus 4pt minus 2pt}{0pt plus 2pt minus 2pt}
\titlespacing\subsubsection{0pt}{12pt plus 4pt minus 2pt}{0pt plus 2pt minus 2pt}

\hypersetup{colorlinks= true, allcolors=blue}

\setlist[itemize]{noitemsep, topsep=0pt}	
\sectionfont{\fontsize{10}{12}\selectfont}
\newcommand{\tab}[1]{\hspace{.2\textwidth}\rlap{#1}}
\newcommand{\itab}[1]{\hspace{0em}\rlap{#1}}
\pagenumbering{gobble}
\begin{document}
\begin{center}
\begin{tabular}{c}
\large\textbf{Siddarth Jayamoorthy}\\
3210 E John Hinkle Pl., Apt D, Bloomington IN 47408 $|$ Ph: 812-272-7823 $|$
\href{mailto:siddarthjay92@gmail.com}{siddarthjay92@gmail.com}\\

\href{https://www.linkedin.com/in/siddarthjay}{linkedin.com/in/siddarthjay}
$\vert$ \href{https://github.com/agent-jay/}{github.com/agent-jay/}\\

\end{tabular}
\vbox{\hrule width 7in height 1pt}
\end{center}

\section*{EDUCATION}
\noindent
\textbf{Master of Science in Data Science\hfill May 2018}\\
Indiana University, Bloomington\\
\textit{Relevant Coursework:} Elements of Artificial Intelligence, Machine Learning,
Introduction to Statistics, \\ 
Algorithm Design and Implementation,
Research Methods and Probabilistic Programming Languages\\
\textbf{Bachelor of Engineering in Mechanical Engineering\hfill May 2014}\\
PSG College of Technology, Tamilnadu, India\\
\textit{Relevant Coursework:} Stochastic Models, Operations Research, Linear
Algebra, Introduction to Computer Science

\section*{PROFESSIONAL EXPERIENCE}
\noindent
\textit{Trainee Decision Scientist, Innovation \& Development Labs, Mu Sigma Pvt. Ltd.\hfill June 2015 - March 2016}
\begin{itemize}
	\item Led a 3 member team to research, prototype and test Computer Vision algorithms for human motion
tracking, as part of a video analytics solution created for a major US retailer
	\item Developed and maintained a multi-user, distributed testing framework for measuring the accuracy of object
detection algorithms, and coupled it with an automated report for consumption by stakeholders
	\item Integrated 2 Internet-of-Things prototypes based on the RaspberryPi platform, with an in-house streaming agent-based platform with quick turnaround in preparation for Scipy 2015, a popular tech conference
\end{itemize}
\textit{Trainee Decision Scientist, Client Services, Mu Sigma Pvt. Ltd \hfill July 2014 - June 2015}
\begin{itemize}
	\item Assisted the analytics division of a Fortune-500 Insurance firm develop and maintain predictive models to detect fraud. Used Text Mining and Ensemble Learning for improved accuracy
	\item Spearheaded a project to create an analytical dataset comprised of 3 years worth of claims. Resolved several data discrepancies which led to 
        overall improvements in the data collection process
    \item Led a 2 member team to integrate third-party information with in-house data. Shaved 70\% of the time taken
to parse the complex structure of the data by automating the process with Python
	\item Ensured consistent delivery of model runs across 3 LOBs and quick turnaround of ad-hoc analyses
	\item Completed a time-critical port of existing production SAS codes from Windows to Unix, creating a
comprehensive catalog of the differences which went on to be used for other projects
	\item Picked up new techniques unfamiliar to the team, and applied them under tight deadlines- web scraping,
topic modeling, geocoding and fuzzy matching
	\item Conducted training sessions for new team members on Python and its data science stack
\end{itemize}


\section*{TECHNICAL SKILLS}
\begin{itemize}
	\item\itab{Techniques} \tab{Predictive modeling, Ensemble learning, Dimensionality reduction, Text Mining}
	\item\itab{Languages} \tab{\textbf{Proficient:} Python, R, SQL; \textbf{Familiar:} SAS, Matlab, Haskell, C/C++}
	\item\itab{Databases} \tab{Postgres, Redis}
	\item\itab{Packages} \tab{Theano, Keras, Scikit-learn, Numpy, pandas, Jupyter, OpenCV, NLTK, Git}
	\item\itab{Platforms} \tab{Linux, Windows, Arduino, RaspberryPi}
	\item\itab{Miscellaneous} \tab{Adobe Photoshop, Microsoft Excel, Microsoft Word}
\end{itemize}


\section*{PROJECTS}
\noindent
\textit{Adversarial game playing engine for Connect 4 \hfill August 2016 - December 2016}
\begin{itemize}
\item Built a strong AI for the Connect 4 board game using Minimax search with Alpha-beta pruning and optimizations such as the Killer
    heuristic, Zobrist hashing and bitboards. Built using \textit{Python}
\end{itemize}
\textit{Hyperparameter optimization for Scikit-learn models\hfill  February 2015 - March 2015}
\begin{itemize}
\item Built a wrapper around Scikit-learn's API to perform hyperparameter
    optimization- Grid and Random search, and produce formatted tables with the
    user's choice of metrics \textit{Python}, \textit{Scikit-learn}
\end{itemize}
\textit{Real-time Map building for mobile robotics using Ultrasound \hfill April 2012 - October 2012}
\begin{itemize}
	\item Created a sensor system for use in mobile robotics to scan the surrounding environment using SONAR and
plot the same on a digital map, in real-time. Built using \textit{Arduino}, \textit{NI LabVIEW}.
\end{itemize}
\textit{Created a clone of Atari Pong\hfill October 2011 - November 2011}
\begin{itemize}
	\item Built a clone of Atari Pong using Object Oriented Design principles. Built using \textit{Python}, \textit{pygame}.
\end{itemize}

\section*{ACHIEVEMENTS}
\begin{itemize}
    \item \textbf{Iterated Prisoner's Dilemma competition}, Elements of AI\hfill First Place
    \item \textbf{Connect 4 competition}, Elements of AI\hfill Second place
	\item \textbf{Probabilistic Graphical Models}, Coursera\hfill Completed
	\item \textbf{Machine Learning}, Coursera\hfill Completed
	\item \textbf{Introduction to Computer Science}, Udacity\hfill Passed with Highest Distinction
\end{itemize}
\end{document}
